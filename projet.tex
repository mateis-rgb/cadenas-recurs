\documentclass{article}

\usepackage[utf8]{inputenc}
\usepackage{amsfonts}
\usepackage{tabto}

\title{Numériques et Sciences Informatiques \\ Projet de groupe n°1}
\author{Augustin B. - Victor H. - Matéis R.}
\date{Octobre - Novembre 2022}

\begin{document}
    \maketitle
    
    La recherche informatique peut se résumer en plusieurs grands axes, mais un des principale est de trouver des solutions à des problèmes. Avec le problème du coffre-fort, il faut résonner en programmation orientée $objet$. Ce grand principe de l'informatique moderne consiste à définir et à interagir avec des briques logicielles appeler objets. Cependant, c'est au développeur d'implémenter ses propres objets en fonction de ses besoins. Il possède une logique un fonctionnement parfois inconnu pour l'utilisateur ce qui permet de définir ce paradigme de programmation comme abstrait. Il y a bien des avantages à programmé avec des objets, mais c'est une manière totalement différente de penser la programmation du problème.
    Dans notre cas, il faut résoudre le faite que 



\end{document}
