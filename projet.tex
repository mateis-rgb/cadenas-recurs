\documentclass{article}

\usepackage[utf8]{inputenc}
\usepackage{amsfonts}
\usepackage{tabto}

\title{Numériques et Sciences Informatiques \\ Projet de groupe n°1}
\author{Augustin B. - Victor H. - Matéis R.}
\date{Octobre - Novembre 2022}

\begin{document}
    \maketitle
    
    La recherche informatique peut se résumer en plusieurs grands axes, mais un des principale est de trouver des solutions à des problèmes. Avec le problème du coffre-fort, il faut résonner en programmation orientée $objet$. Ce grand principe de l'informatique moderne consiste à définir et à interagir avec des briques logicielles appeler objets. Cependant, c'est au développeur d'implémenter ses propres objets en fonction de ses besoins. Il possède une logique un fonctionnement parfois inconnu pour l'utilisateur, ce qui permet de définir ce paradigme de programmation comme abstrait. Il y a bien des avantages à programmer avec des objets, mais c'est une manière totalement différente de penser la programmation du problème. \newline
    
    Dans notre cas, il faut résoudre un problème de coffre fort avec des combinaisons. Ce jeu ce joue à plusieurs joueurs avec autant de cadenas que de participants. Le but de ce jeu est de trouver la combinaison de ce coffre avant les autres joueurs, et en fonction de qui trouve la combinaison en premier obtient une récompense et ainsi de suite. Dans le cas où il n'y à qu'un seul joueur, ce participant est limiter en nombre d'essais. \newline
    
    Quand on lance le jeu, après avoir indiquer le nombre de joueurs, des combinaisons de 5 caractère se génère aléatoirement a raison de une combinaison par joueurs. Lorsque l'on essaie de rentrer une combinaison, le programme nous indique s'il y a des valeurs de l'essaie présente dans la combinaison. \newline


\end{document}
